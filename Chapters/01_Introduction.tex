
\section{Background}
The US military relies on airlift to not only deploy and sustain U.S. armed forces anywhere in the world but also to rapidly mobilize humanitarian efforts and supplies.  A growing concern for the modern military budget is how to provide airlift functions expediently with economical awareness. Research on the subject of cargo transportation via aircraft has existed as long as the advent of military airlift itself. With it, a myriad of paths to improve cargo transportation have emerged.  Historically, to improve cargo throughput, aircraft designers emphasized larger and faster planes, however recently these efforts have plateaued.  New planes are marginally better than old in the realm of size and speed.  %The fleet of the United States Air Force's main cargo airlift provider, Air Mobility Command (AMC), is aging.  Currently, there are 52 C-5B/C/M, the newest of which was delivered to the Air Force in 1989 \cite{C5}.  Military expenditure in the near future will not be towards a new air mobility fleet (inferred from old DoD "The Mobility Capabilities and Requirements Study 2016 (MCRS-16)" published in 2009 new ). 
Recent considerations of the improvement of air transportation have delved into fuel efficiency. Reiman proposed improving fuel efficiency and cargo throughput through alternative routing methods \cite{Reiman2014}. Boone presented a methodology that incorporates ensemble, versus deterministic, numerical weather prediction models into route planning, thus reducing the
amount of excess fuel burned by poor forecasts and providing a range of potential values which aid in flight planning \cite{Boone2018}.    

\section{Motivation}

Optimization has been widely used in the civilian sector to save costs, promote efficiency, and reduce waste.  While government entities should endeavor to optimize for the same reasons, they also utilize optimization to conserve manpower, improve lethality, and save lives. Supplying troops and humanitarian aid is a military transportation problem.  %Travel on nonproductive legs diminishes the aircraft's productivity, and more often than not, cargo aircraft must fly one ormore positioning legs to an onload location \cite{AFPAM10-1403}.  
Operations already impacted by the limited capacity of aircraft also fall prey to dynamic requirements and differing priorities of multiple global locations.  The question is how to assign and utilize varying cargo aircraft types to these demands to minimize cost and maximize priority fulfillment.  %The inherent tradeoff is fulfilling demands now, with the possible loss of productivity of an aircraft, or waiting for future demands thus maximizing the productivity of the aircraft but not fulfilling all demands expediently. The priority of the demand, positioning of assets, capacity of aircraft, and transportation cost all affect this decision.
Solving the problem can be done through modeling the problem as an extension of the traditional knapsack problem and the assignment problem.  Chapter 2 reviews current methodologies for solving the multiple knapsack assignment problem as well as the foundational research to the all-inclusive approach to airlift planning in the Aircraft Selection Model (ASM) \cite{maywald}.

%``Fuel efficiency must be assessed simultaneously with cargo throughput which is the primary goal of airlift effectiveness." \cite{Reiman}

